\documentclass[11pt]{article}
\usepackage{setspace}
\setstretch{1}
\usepackage{amsmath,amssymb, amsthm}
\usepackage{graphicx}
\usepackage{bm}
\usepackage[hang, flushmargin]{footmisc}
\usepackage[colorlinks=true]{hyperref}
\usepackage[nameinlink]{cleveref}
\usepackage{footnotebackref}
\usepackage{url}
\usepackage{listings}
\usepackage[most]{tcolorbox}
\usepackage{inconsolata}
\usepackage[papersize={8.5in,11in}, margin=1in]{geometry}
\usepackage{float}
\usepackage{caption}
\usepackage{esint}
\usepackage{url}
\usepackage{enumitem}
\usepackage{subfig}
\usepackage{wasysym}
\newcommand{\ilc}{\texttt}
\usepackage{etoolbox}
\usepackage{algorithm}
\usepackage{changepage}
% \usepackage{algorithmic}
\usepackage[noend]{algpseudocode}
\usepackage{tikz}
\usepackage{pifont}
\usepackage{gensymb}
\usetikzlibrary{matrix,positioning,arrows.meta,arrows}
\patchcmd{\thebibliography}{\section*{\refname}}{}{}{}
% \PassOptionsToPackage{hyphens}{url}\usepackage{hyperref}

\providecommand{\myceil}[1]{\left \lceil #1 \right \rceil }
\providecommand{\myfloor}[1]{\left \lfloor #1 \right \rfloor }
\providecommand{\qbm}[1]{\begin{bmatrix} #1 \end{bmatrix}}
\providecommand{\qpm}[1]{\begin{pmatrix} #1 \end{pmatrix}}
\providecommand{\norm}[1]{\left\lVert #1 \right\rVert}
\providecommand{\len}[1]{\left| #1 \right|}
\newcommand{\cmark}{\ding{51}}%
\newcommand{\xmark}{\ding{55}}%

\begin{document}



\title{\textbf{MATH 307: Individual Homework 19}}


\author{Shaochen (Henry) ZHONG, \ilc{sxz517@case.edu}}

\date{Due and submitted on 04/19/2021 \\ Spring 2021, Dr. Guo}
\maketitle



\subsection*{Problem 1}
\textit{See HW instruction.}\newline

We start from the identity matrix and follow the operations on the instruction:

\begin{align*}
    E &= \qbm{1 & 0 & 0 & 0 \\
        0 & 1 & 0 & 0 \\
        0 & 2 & 1 & 0 \\
        0 & 0 & 0 & 1 \\}
    \qbm{0 & 0 & 0 & 1 \\
        0 & 1 & 0 & 0 \\
        0 & 0 & 1 & 0 \\
        1 & 0 & 0 & 0 \\}
    \qbm{1 & 0 & 0 & 0 \\
        0 & -3 & 0 & 0 \\
        0 & 0 & 1 & 0 \\
        0 & 0 & 0 & 1 \\} \\
    &= \qbm{1 & 0 & 0 & 0 \\
        0 & 1 & 0 & 0 \\
        0 & 2 & 1 & 0 \\
        0 & 0 & 0 & 1 \\}
        \qbm{0 & 0 & 0 & 1 \\
            0 & -3 & 0 & 0 \\
            0 & 0 & 1 & 0 \\
            1 & 0 & 0 & 0 \\} \\
    &= \qbm{0 & 0 & 0 & 1 \\
        0 & -3 & 0 & 0 \\
        0 & -6 & 1 & 0 \\
        1 & 0 & 0 & 0 \\}
\end{align*}

Now for $E^{-1}$, we have:

\begin{align*}
    E^{-1} &= \qbm{1 & 0 & 0 & 0 \\
        0 & -\frac{1}{3} & 0 & 0 \\
        0 & 0 & 1 & 0 \\
        0 & 0 & 0 & 1 \\}
        \qbm{0 & 0 & 0 & 1 \\
        0 & 1 & 0 & 0 \\
        0 & 0 & 1 & 0 \\
        1 & 0 & 0 & 0 \\}
        \qbm{1 & 0 & 0 & 0 \\
        0 & 1 & 0 & 0 \\
        0 & -2 & 1 & 0 \\
        0 & 0 & 0 & 1 \\} \\
        &= \qbm{1 & 0 & 0 & 0 \\
            0 & -\frac{1}{3} & 0 & 0 \\
            0 & 0 & 1 & 0 \\
            0 & 0 & 0 & 1 \\}
            \qbm{0 & 0 & 0 & 1 \\
        0 & 1 & 0 & 0 \\
        0 & -2 & 1 & 0 \\
        1 & 0 & 0 & 0 \\} \\
        &= \qbm{0 & 0 & 0 & 1 \\
        0 & -\frac{1}{3} & 0 & 0 \\
        0 & -2 & 1 & 0 \\
        1 & 0 & 0 & 0 \\}
\end{align*}

To verify we have:

\begin{equation*}
    E E^{-1} = \qbm{0 & 0 & 0 & 1 \\
        0 & -3 & 0 & 0 \\
        0 & -6 & 1 & 0 \\
        1 & 0 & 0 & 0 \\}
        \qbm{0 & 0 & 0 & 1 \\
    0 & -\frac{1}{3} & 0 & 0 \\
    0 & -2 & 1 & 0 \\
    1 & 0 & 0 & 0 \\} = I
\end{equation*}

\subsection*{Problem 2}
\textit{See HW instruction.}\newline

For a matrix to be in \textit{rref} it must have the following properties (\href{https://www.math.fsu.edu/~bellenot/class/f08/lalab/other/rref2.pdf}{ref.}):

\begin{enumerate}
    \item The first non-zero entry in any row is the number 1, these are called pivots. (So each row can have zero or one pivot.)
    \item A pivot is the only non-zero entry in its column. (So each column can have zero or one pivot.)
    \item Rows are orders so that rows of all zeros are at the bottom, and the pivots are in column order.
\end{enumerate}

Now to evaluate the given matrices with respect to the above rules:

\begin{itemize}
    \item \textbf{A}: Failed rule 1.
    \item \textbf{B}: Failed rule 3 (pivots not in column order).
    \item \textbf{C}: Failed rule 2.
    \item \textbf{D}: It is in rref form.
\end{itemize}


\end{document}

