\documentclass[11pt]{article}
\usepackage{setspace}
\setstretch{1}
\usepackage{amsmath,amssymb, amsthm}
\usepackage{graphicx}
\usepackage{bm}
\usepackage[hang, flushmargin]{footmisc}
\usepackage[colorlinks=true]{hyperref}
\usepackage[nameinlink]{cleveref}
\usepackage{footnotebackref}
\usepackage{url}
\usepackage{listings}
\usepackage[most]{tcolorbox}
\usepackage{inconsolata}
\usepackage[papersize={8.5in,11in}, margin=1in]{geometry}
\usepackage{float}
\usepackage{caption}
\usepackage{esint}
\usepackage{url}
\usepackage{enumitem}
\usepackage{subfig}
\usepackage{wasysym}
\newcommand{\ilc}{\texttt}
\usepackage{etoolbox}
\usepackage{algorithm}
\usepackage{changepage}
% \usepackage{algorithmic}
\usepackage[noend]{algpseudocode}
\usepackage{tikz}
\usepackage{pifont}
\usepackage{gensymb}
\usetikzlibrary{matrix,positioning,arrows.meta,arrows}
\patchcmd{\thebibliography}{\section*{\refname}}{}{}{}
% \PassOptionsToPackage{hyphens}{url}\usepackage{hyperref}

\providecommand{\myceil}[1]{\left \lceil #1 \right \rceil }
\providecommand{\myfloor}[1]{\left \lfloor #1 \right \rfloor }
\providecommand{\qbm}[1]{\begin{bmatrix} #1 \end{bmatrix}}
\providecommand{\qpm}[1]{\begin{pmatrix} #1 \end{pmatrix}}
\providecommand{\norm}[1]{\left\lVert #1 \right\rVert}
\providecommand{\len}[1]{\left| #1 \right|}
\newcommand{\cmark}{\ding{51}}%
\newcommand{\xmark}{\ding{55}}%

\begin{document}



\title{\textbf{MATH 307: Individual Homework 21}}


\author{Shaochen (Henry) ZHONG, \ilc{sxz517@case.edu}}

\date{Due and submitted on 04/28/2021 \\ Spring 2021, Dr. Guo}
\maketitle



\subsection*{Problem 1}
\textit{See HW instruction.}\newline

Know that $L_1 = I + \ell_1 e^T_1$ and W.T.S. that $L_1^{-1} = I - \ell_1 e^T_1$, we may simply check if $L_1 (I - \ell_1 e^T_1) = I$.

\begin{align*}
    L (I - \ell_1 e^T_1) &=  (I + \ell_1 e^T_1) (I - \ell_1 e^T_1) = I - \ell_1 e^T_1 + \ell_1 e^T_1 - \ell_1 e^T_1 \ell_1 e^T_1 \\
    &= I - \ell_1 (e^T_1 \ell_1) e^T_1 \\
    &= I - \ell_1 (\qbm{1 & 0 & \dots & 0} \qbm{0 \\ -\frac{a_{21}}{a_{11}} \\ \vdots \\ -\frac{a_{n1}}{a_{11}} }) \\
    &= I - ell_1 (0) e^T_1 \\
    &= I
\end{align*}

As $L (I - \ell_1 e^T_1) = I$, there must be $L_1^{-1} = I - \ell_1 e^T_1$ as $L_1 L_1^{-1} = I$.


\subsection*{Problem 2}
\textit{See HW instruction.}\newline

Start with Gaussian elimination:

\begin{align*}
    &\underbrace{\qbm{1 & 2 & 1 & 3 & \vline & 2 \\
        -3 & 2 & 1 & 0 & \vline & -5 \\
        3 & 2 & 1 & 1 & \vline & 2}}_{R_2 + 3R_1 \to R_2, \ R_3 - 3R_1 \to R_3} =
    \underbrace{\qbm{1 & 2 & 1 & 3 & \vline & 2 \\
        0 & 8 & 4 & 9 & \vline & 1 \\
        0 & -4 & -2 & -8 & \vline & -4}}_{R_3 + \frac{R_2}{2} \to R_3} =
    \underbrace{\qbm{1 & 2 & 1 & 3 & \vline & 2 \\
        0 & 8 & 4 & 9 & \vline & 1 \\
        0 & 0 & 0 & -\frac{7}{2} & \vline & -\frac{7}{2}}}_{-\frac{2}{7}R_3 \to R_3} =
    \qbm{1 & 2 & 1 & 3 & \vline & 2 \\
        0 & 8 & 4 & 9 & \vline & 1 \\
        0 & 0 & 0 & 1 & \vline & 1}
\end{align*}


    % \underbrace{\qbm{1 & 2 & 1 & 3 & \vline & 2 \\
    %     0 & 1 & \frac{1}{2} & \frac{9}{8} & \vline & \frac{1}{8} \\
    %     0 & -4 & -2 & -8 & \vline & -4}}_{R_1 - 2R_2 \to R_1,\  R_3 + 4R_2 \to R_3}\\ &=
    % \underbrace{\qbm{1 & 0 & 0 & 0.75 & \vline & 1.75 \\
    %     0 & 1 & 0.5 & 1.125 & \vline & 0.125 \\
    %     0 & 0 & 0 & -3.5 & \vline & -3.5}}_{\frac{R_3}{-3.5} \to R_3} =
    % \underbrace{\qbm{1 & 0 & 0 & 0.75 & \vline & 1 \\
    %     0 & 1 & 0.5 & 1.125 & \vline & 0.125 \\
    %     0 & 0 & 0 & 1 & \vline & 1}}_{R_1 - 0.75 R_3 \to R_1, \ R_2 - 1.125 R_3 \to R_2} =
    % \qbm{1 & 0 & 0 & 0 & \vline & 1 \\
    %     0 & 1 & 0.5 & 0 & \vline & -1 \\
    %     0 & 0 & 0 & 1 & \vline & 1}

So we have a system of:

\begin{equation*}
    \begin{cases}
        x_1 + 2x_2 + x_3 + 3x_4= 2 \\
        8x_2 + 4 x_3 + 9 x_4 = 1 \\
        x_4 = 1
    \end{cases}
\end{equation*}

We have the first, second, and fourth columns being the pivot columns. Let the free column $x_3 = 0$, we have:

\begin{align*}
    &\begin{cases}
        x_1 + 2x_2 + 0 + 3 = 2 \\
        8x_2 + 0 + 9 = 1 \\
    \end{cases} \\
    \Longrightarrow &\begin{cases}
        x_1 = 1 \\
        x_2 = -1 \\
    \end{cases}
\end{align*}

Thus, $x_p = \qbm{1 \\ -1 \\ 0 \\ 1}$. Now to find $null(A)$, which we know is same as $ null(\qbm{1 & 2 & 1 & 3  \\
    0 & 8 & 4 & 9  \\
    0 & 0 & 0 & 1 })$. We let the $x_3 = 1$ and have:


    \begin{align*}
        &\begin{cases}
            x_1 + 2x_2 + 1  = 0 \\
            8x_2 + 4   = 0 \\
            x_4 = 0
        \end{cases} \\
        \Longrightarrow &\begin{cases}
            x_1 = 0 \\
            x_2 = -\frac{1}{2} \\
        \end{cases}
    \end{align*}

Thus, we have $null(A) = \qbm{0 \\ -\frac{1}{2} \\ 1 \\ 0}$; which we may confirm by checking $null(A) + rank(A) = dim(A) \Longleftrightarrow 1 + 3 = 4$. And we may there have the general solution as $x_g =  \qbm{1 \\ -1 \\ 0 \\ 1} + \alpha \qbm{0 \\ -\frac{1}{2} \\ 1 \\ 0}$.


\subsection*{Problem 2}
\textit{See HW instruction.}\newline

\subsubsection*{(a)}

\begin{align*}
    &\underbrace{\qbm{-1 & 2 & 1 & 0 & 2 & \vline & -1 \\
        2 & 0 & 0 & 3 & -1 & \vline & 0 \\
        -1 & 6 & 3 & 3 & 5 & \vline & c}}_{2R_1 + R_2 \to R_2, \ -R_1 + R_3 \to R_3} =
    \qbm{-1 & 2 & 1 & 0 & 2 & \vline & -1 \\
        0 & 4 & 2 & 3 & 3 & \vline & -2 \\
        0 & 4 & 2 & 3 & 3 & \vline & c + 1 = -2 \Rightarrow c = -3}
\end{align*}

So we must have $c = -3$ as otherwise there must be a conflict between $R_2$ and $R_3$.

\subsubsection*{(b)}

To find $x_p$ we continue the row deduction:

\begin{align*}
    \underbrace{\qbm{-1 & 2 & 1 & 0 & 2 & \vline & -1 \\
        0 & 4 & 2 & 3 & 3 & \vline & -2 \\
        0 & 4 & 2 & 3 & 3 & \vline & -2}}_{-R_2 + R_3 \to R_3} =
        \qbm{-1 & 2 & 1 & 0 & 2 & \vline & -1 \\
    0 & 4 & 2 & 3 & 3 & \vline & -2 \\
    0 & 0 & 0 & 0 & 0 & \vline & 0}
\end{align*}

By inspecting the reduced matrix we know that columns 3, 4, 5 are all free variables, thus we may have:

\begin{align*}
    &\begin{cases}
        -x_1 + 2x_2 + x_3 + 2x_5 = -1 \\
        4x_2 + 2x_3 + 3x_4 + 3x_5 = -2
    \end{cases} \\
    &\text{Let} \ x_3 = x_4 =x_5 = 0 \\
    \Longrightarrow &\begin{cases}
        -x_1 + 2x_2 = -1 \Longleftrightarrow -x_1 - 1 = -1 & x_1 = 0 \\
        4x_2 = -2 & x_2 = -\frac{1}{2}
    \end{cases}
\end{align*}

Thus, we may have $x_p = \qbm{1 \\ -\frac{1}{2} \\ 0 \\ 0 \\ 0}$.




\end{document}

