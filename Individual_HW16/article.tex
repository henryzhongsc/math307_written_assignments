\documentclass[11pt]{article}
\usepackage{setspace}
\setstretch{1}
\usepackage{amsmath,amssymb, amsthm}
\usepackage{graphicx}
\usepackage{bm}
\usepackage[hang, flushmargin]{footmisc}
\usepackage[colorlinks=true]{hyperref}
\usepackage[nameinlink]{cleveref}
\usepackage{footnotebackref}
\usepackage{url}
\usepackage{listings}
\usepackage[most]{tcolorbox}
\usepackage{inconsolata}
\usepackage[papersize={8.5in,11in}, margin=1in]{geometry}
\usepackage{float}
\usepackage{caption}
\usepackage{esint}
\usepackage{url}
\usepackage{enumitem}
\usepackage{subfig}
\usepackage{wasysym}
\newcommand{\ilc}{\texttt}
\usepackage{etoolbox}
\usepackage{algorithm}
\usepackage{changepage}
% \usepackage{algorithmic}
\usepackage[noend]{algpseudocode}
\usepackage{tikz}
\usepackage{pifont}
\usepackage{gensymb}
\usetikzlibrary{matrix,positioning,arrows.meta,arrows}
\patchcmd{\thebibliography}{\section*{\refname}}{}{}{}
% \PassOptionsToPackage{hyphens}{url}\usepackage{hyperref}

\providecommand{\myceil}[1]{\left \lceil #1 \right \rceil }
\providecommand{\myfloor}[1]{\left \lfloor #1 \right \rfloor }
\providecommand{\qbm}[1]{\begin{bmatrix} #1 \end{bmatrix}}
\providecommand{\qpm}[1]{\begin{pmatrix} #1 \end{pmatrix}}
\providecommand{\norm}[1]{\left\lVert #1 \right\rVert}
\providecommand{\len}[1]{\left| #1 \right|}
\newcommand{\cmark}{\ding{51}}%
\newcommand{\xmark}{\ding{55}}%

\begin{document}



\title{\textbf{MATH 307: Individual Homework 16}}


\author{Shaochen (Henry) ZHONG, \ilc{sxz517@case.edu}}

\date{Due and submitted on 04/05/2021 \\ Spring 2021, Dr. Guo}
\maketitle



\subsection*{Problem 1}
\textit{See HW instruction.}\newline

First to find eigenvalues:
\begin{align*}
    \det(A - \lambda I) &= \det\qbm{-2-\lambda & 2 \\ -1 & -3-\lambda} = 0 \\
    0 &= (-2-\lambda)(-3-\lambda) - (-2 \cdot -1)  \\
    &= \lambda^2 + 5\lambda + 4 = (\lambda + 1)(\lambda + 4) \\
    \Longrightarrow &\begin{cases}
        \lambda_1 = -1 \\
        \lambda_2 = -4
    \end{cases}
\end{align*}

Then to find the corresponding eigenvectors:
\begin{align*}
    Av_1 &= \lambda_1 v_1 \\
    \qbm{-2x - 2y \\ -1x -3y} &= \qbm{\lambda_1 x \\ \lambda_1 y} = \qbm{-x \\ -y} \\
    x &= -2y \\
    \Longrightarrow v_1 &= \qbm{-2 \\ 1}
\end{align*}

\begin{align*}
    Av_2 &= \lambda_2 v_2 \\
    \qbm{-2x - 2y \\ -1x -3y} &= \qbm{\lambda_2 x \\ \lambda_2 y} = \qbm{-4x \\ -4y} \\
    2x &= 2y \\
    \Longrightarrow v_2 &= \qbm{1 \\ 1}
\end{align*}

Thus, we have eigenvalues being $\lambda_1 = -1, \lambda_2 = -4$ and their corresponding eigenvectors being $\qbm{-2 \\ 1}, \qbm{1 \\ 1}$.

\subsection*{Problem 2}
\textit{See HW instruction.}\newline

% https://math.stackexchange.com/questions/767835/proving-eigenvalue-squared-is-eigenvalue-of-a2/767838

First we want to find out what will be the eigenvalue and its corresponding eigenvector for $A^-1$

\begin{align*}
    Av &= \lambda v \\
    A^{-1} Av &= A^{-1} \lambda v \\
    v &= A^{-1} \lambda v \\
    \lambda^{-1} v &= \lambda^{-1} A^{-1} \lambda v \\
    \lambda^{-1} v &= Av
\end{align*}

Known that the eigenvalue for $A^{-1}$ is $\lambda^{-1}$ and its corresponding eigenvector is still $v$. Assume that $Bv = \lambda v$ we want to show $B^k$ has eigenvalue $\lambda^k$ and eigenvector $v$.

\begin{align*}
    B^{k-1} Bv &= B^{k-1} \lambda v \\
    &= B^{k-2} \lambda (Bv) = B^{k-2} \lambda (\lambda v) \\
    B^k v &= B^{k-2} \lambda^2 v \\
    &= B^{k-3} \lambda^2 (Bv) = B^{k-3} \lambda^3 v \\
    &= B^{k-4} \lambda^3 (Bv) = B^{k-4} \lambda^4 v \\
    &\dots \\
    B^k v &= B^{k-k} \lambda^{k-1} (Bv) = \lambda^k v
\end{align*}

So in this case we have $B = A^-1$ and $k = 3$, so we have $(A^{-1})^3 v = (\lambda^{-1})^3 v$. As $(\lambda^{-1})^3 = \lambda^{-3}$ is the eigenvalue and $v$ is the corresponding eigenvector for $(A^{-1})^3$.


\subsection*{Problem 3}
\textit{See HW instruction.}\newline

In the previous $\textbf{Problem 2}$ we have established that $Av = \lambda v \Longleftrightarrow A^k v = \lambda^k v$, so we must have $Pv = \lambda v \Longleftrightarrow P^2 v = \lambda^2 v$.

Since $P = P^2$, we have $Pv = P^2 v = \lambda ^2 v = \lambda v$, which implies $\lambda ^2 = \lambda$. Thus, there must be $\lambda = \{0, 1\}$.



\subsection*{Problem 4}
\textit{See HW instruction.}\newline




\end{document}

