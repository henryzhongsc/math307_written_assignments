\documentclass[11pt]{article}
\usepackage{setspace}
\setstretch{1}
\usepackage{amsmath,amssymb, amsthm}
\usepackage{graphicx}
\usepackage{bm}
\usepackage[hang, flushmargin]{footmisc}
\usepackage[colorlinks=true]{hyperref}
\usepackage[nameinlink]{cleveref}
\usepackage{footnotebackref}
\usepackage{url}
\usepackage{listings}
\usepackage[most]{tcolorbox}
\usepackage{inconsolata}
\usepackage[papersize={8.5in,11in}, margin=1in]{geometry}
\usepackage{float}
\usepackage{caption}
\usepackage{esint}
\usepackage{url}
\usepackage{enumitem}
\usepackage{subfig}
\usepackage{wasysym}
\newcommand{\ilc}{\texttt}
\usepackage{etoolbox}
\usepackage{algorithm}
\usepackage{changepage}
% \usepackage{algorithmic}
\usepackage[noend]{algpseudocode}
\usepackage{tikz}
\usepackage{gensymb}
\usetikzlibrary{matrix,positioning,arrows.meta,arrows}
\patchcmd{\thebibliography}{\section*{\refname}}{}{}{}
% \PassOptionsToPackage{hyphens}{url}\usepackage{hyperref}

\providecommand{\myceil}[1]{\left \lceil #1 \right \rceil }
\providecommand{\myfloor}[1]{\left \lfloor #1 \right \rfloor }
\providecommand{\qbm}[1]{\begin{bmatrix} #1 \end{bmatrix}}
\providecommand{\qpm}[1]{\begin{pmatrix} #1 \end{pmatrix}}
\providecommand{\norm}[1]{\left\lVert #1 \right\rVert}
\providecommand{\len}[1]{\left| #1 \right|}

\begin{document}




\title{\textbf{MATH 307: Group Homework 10}}


\author{\textit{Group 8}\\
Shaochen (Henry) ZHONG, Zhitao (Robert) CHEN, John MAYS, Huaijin XIN\\ \ilc{\{sxz517, zxc325, jkm100, hxx200\}@case.edu}}

\date{Due and submitted on 04/26/2021 \\ Spring 2021, Dr. Guo}
\maketitle



\subsection*{Problem 1}
\textit{See HW instruction.}\newline

\subsubsection*{(a) LU decomposition}

\begin{align*}
\qbm{2 & 3 \\ 1 & 4} x &= \qbm{1 \\ 3} & \Rightarrow \qbm{1 & 0 \\ \frac{1}{2} & 1} \qbm{2 & 3 \\ 0 & \frac{5}{2}} x &= \qbm{1 \\ 3} \\
\text{Let} \qbm{2 & 3 \\ 0 & \frac{5}{2}} x &= y \\
\text{We have} \qbm{1 & 0 \\ \frac{1}{2} & 1}  y &= \qbm{1 \\ 3} & \Rightarrow y = \qbm{1 \\ \frac{5}{2}} \\
\Longrightarrow \qbm{2 & 3 \\ 0 & \frac{5}{2}} x &= y = \qbm{1 \\ \frac{5}{2}} & \Rightarrow x = \qbm{-1 \\ 1}
\end{align*}

\subsubsection*{(b) Gaussian elimination}

\begin{align*}
    \underbrace{\qbm{ 2 & 3 & \vline & 1 \\ 1 & 4 & \vline & 3}}_{R_2 - \frac{1}{2} R_1 \ \to R_2} &= \qbm{ 2 & 3 & \vline & 1 \\ 0 & \frac{5}{2} & \vline & \frac{5}{2}}
    \Rightarrow \begin{cases}
        \frac{5}{2} x_2 = \frac{5}{2} & x_2 = 1 \\
        2 x_1 + 3(1) = 1 & x_1 = -1
    \end{cases} \\
    \Longrightarrow x &= \qbm{-1 \\ 1}
\end{align*}



\subsubsection*{(c) Gauss-Jordan elimination}

% https://atozmath.com/CONM/GaussEli.aspx?q=GE1&q1=1%602%2c3%3b1%2c4%60GE1&dm=D&dp=4&do=1#PrevPart

\begin{align*}
    \underbrace{\qbm{ 2 & 3 & \vline & 1 & 0 \\ 1 & 4 & \vline & 0 & 1}}_{\frac{1}{2} R_1 \ \to R_1} &= \underbrace{\qbm{ 1 & \frac{3}{2} & \vline & \frac{1}{2} & 0 \\ 1 & 4 & \vline & 0 & 1}}_{R_2 - R_1 \ \to R_2} = \underbrace{\qbm{ 1 & \frac{3}{2} & \vline & \frac{1}{2} & 0 \\ 0 & \frac{5}{2} & \vline & -\frac{1}{2} & 1}}_{\frac{5}{2}R_2 \ \to R_2} = \underbrace{\qbm{ 1 & \frac{3}{2} & \vline & \frac{1}{2} & 0 \\ 0 & 1 & \vline & -\frac{2}{10} & \frac{2}{5}}}_{R_1 - \frac{3}{2} R_2 \ \to R_1} = \qbm{ 1 & 0 & \vline & \frac{8}{10} & -\frac{3}{5} \\ 0 & 1 & \vline & -\frac{2}{10} & \frac{2}{5}} \\
    \Rightarrow A^{-1} &= \qbm{ 1 & 0 & \vline & \frac{8}{10} & -\frac{3}{5} \\ 0 & 1 & \vline & -\frac{2}{10} & \frac{2}{5}} \\
    &\text{Known that} \ x = A^{-1} b \\
    x &= \qbm{ 1 & 0 & \vline & \frac{8}{10} & -\frac{3}{5} \\ 0 & 1 & \vline & -\frac{2}{10} & \frac{2}{5}} \qbm{1 \\ 3} = \qbm{\frac{8}{10} - \frac{18}{10} \\ -\frac{2}{10} + \frac{12}{10}} = \qbm{-1 \\ 1}
\end{align*}\newline

\noindent To verify, we have $Ax = \qbm{2 & 3 \\ 1 & 4} \qbm{-1 \\ 1} = \qbm{ -2 + 3 \\ -1 + 4} = \qbm{1 \\ 3} = b$.

\subsection*{Problem 2}
\textit{See HW instruction.}\newline

We omitted the steps to find out this $rref(A)$ as it is too much work to show in \LaTeX, the finding is:

\begin{align*}
    rref(A) &= \qbm{\mathbf{1} & 0 & 0 & 0 \\ 0 & \mathbf{1} & \frac{1}{2} & 0 \\ 0 & 0 & 0 & \mathbf{1}}
\end{align*}

So the three pivots are the \textbf{bold 1} showed above. The rank of $A$ is the number of pivots, thus $rank(A) = 3$. And per the rank-nullity theorem, we have the nullity $dim(A) - rank(A) = 4 - 3 = 1$.

Now for column space, since we have pivots on the fisrt, second, and fourth columns, we have $range(A) = span(\qbm{ 1\\ -3 \\ 3} \qbm{2 \\ 2\\ 2} \qbm{3 \\ 0 \\1})$. And for nullspace, let $Ax = 0$, we have :

\begin{align*}
    &\begin{cases}
        x_1 = 0  \\
        x_2 + \frac{1}{2} x_3 = 0 \\
        x_4 = 0
    \end{cases} \\
    \qbm{x_1 \\ x_2 \\ x_3 \\ x_4} &= \qbm{0 \\ x_2 \\ -2 x_2 \\ 0} = r\qbm{0 \\ 1 \\ -2  \\ 0} \\
    null(A) &= span(\qbm{0 \\ 1 \\ -2  \\ 0})
\end{align*}




\end{document}

