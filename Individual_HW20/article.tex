\documentclass[11pt]{article}
\usepackage{setspace}
\setstretch{1}
\usepackage{amsmath,amssymb, amsthm}
\usepackage{graphicx}
\usepackage{bm}
\usepackage[hang, flushmargin]{footmisc}
\usepackage[colorlinks=true]{hyperref}
\usepackage[nameinlink]{cleveref}
\usepackage{footnotebackref}
\usepackage{url}
\usepackage{listings}
\usepackage[most]{tcolorbox}
\usepackage{inconsolata}
\usepackage[papersize={8.5in,11in}, margin=1in]{geometry}
\usepackage{float}
\usepackage{caption}
\usepackage{esint}
\usepackage{url}
\usepackage{enumitem}
\usepackage{subfig}
\usepackage{wasysym}
\newcommand{\ilc}{\texttt}
\usepackage{etoolbox}
\usepackage{algorithm}
\usepackage{changepage}
% \usepackage{algorithmic}
\usepackage[noend]{algpseudocode}
\usepackage{tikz}
\usepackage{pifont}
\usepackage{gensymb}
\usetikzlibrary{matrix,positioning,arrows.meta,arrows}
\patchcmd{\thebibliography}{\section*{\refname}}{}{}{}
% \PassOptionsToPackage{hyphens}{url}\usepackage{hyperref}

\providecommand{\myceil}[1]{\left \lceil #1 \right \rceil }
\providecommand{\myfloor}[1]{\left \lfloor #1 \right \rfloor }
\providecommand{\qbm}[1]{\begin{bmatrix} #1 \end{bmatrix}}
\providecommand{\qpm}[1]{\begin{pmatrix} #1 \end{pmatrix}}
\providecommand{\norm}[1]{\left\lVert #1 \right\rVert}
\providecommand{\len}[1]{\left| #1 \right|}
\newcommand{\cmark}{\ding{51}}%
\newcommand{\xmark}{\ding{55}}%

\begin{document}



\title{\textbf{MATH 307: Individual Homework 20}}


\author{Shaochen (Henry) ZHONG, \ilc{sxz517@case.edu}}

\date{Due and submitted on 04/21/2021 \\ Spring 2021, Dr. Guo}
\maketitle



\subsection*{Problem 1}
\textit{See HW instruction.}\newline

We will first find $U$ with row reduction.

\begin{equation*}
    U = \underbrace{\qbm{3 & 2 & 1 \\ 6 & 6 & 3 \\ 3 & 0 & -1 \\}}_{R2 - \mathbf{(2)}R1} \Rightarrow \underbrace{\qbm{3 & 2 & 1 \\ 0 & 2 & 1 \\ 3 & 0 & -1 \\}}_{R3 - \mathbf{(1)}R1} \Rightarrow \underbrace{\qbm{3 & 2 & 1 \\ 0 & 2 & 1 \\ 0 & -2 & -2 \\}}_{R3 - \mathbf{(-1)}R2} \Rightarrow \qbm{3 & 2 & 1 \\ 0 & 2 & 1 \\ 0 & 0 & -1 \\}
\end{equation*}

Then we start $L$ from $I$ and add the above operations (highlighted in bold), we have $L = \qbm{1 & 0 & 0 \\ 2 & 1 & 0 \\ 1 & -1 & 1}$.

To verify, we have:

\begin{equation*}
    LU = \qbm{1 & 0 & 0 \\ 2 & 1 & 0 \\ 1 & -1 & 1} \qbm{3 & 2 & 1 \\ 0 & 2 & 1 \\ 0 & 0 & -1 \\} = \qbm{3 & 2 & 1 \\ 6 & 6 & 3 \\ 3 & 0 & -1 \\} = A
\end{equation*}



\end{document}

