\documentclass[11pt]{article}
\usepackage{setspace}
\setstretch{1}
\usepackage{amsmath,amssymb, amsthm}
\usepackage{graphicx}
\usepackage{bm}
\usepackage[hang, flushmargin]{footmisc}
\usepackage[colorlinks=true]{hyperref}
\usepackage[nameinlink]{cleveref}
\usepackage{footnotebackref}
\usepackage{url}
\usepackage{listings}
\usepackage[most]{tcolorbox}
\usepackage{inconsolata}
\usepackage[papersize={8.5in,11in}, margin=1in]{geometry}
\usepackage{float}
\usepackage{caption}
\usepackage{esint}
\usepackage{url}
\usepackage{enumitem}
\usepackage{subfig}
\usepackage{wasysym}
\newcommand{\ilc}{\texttt}
\usepackage{etoolbox}
\usepackage{algorithm}
\usepackage{changepage}
% \usepackage{algorithmic}
\usepackage[noend]{algpseudocode}
\usepackage{tikz}
\usepackage{pifont}
\usetikzlibrary{matrix,positioning,arrows.meta,arrows}
\patchcmd{\thebibliography}{\section*{\refname}}{}{}{}
% \PassOptionsToPackage{hyphens}{url}\usepackage{hyperref}

\providecommand{\myceil}[1]{\left \lceil #1 \right \rceil }
\providecommand{\myfloor}[1]{\left \lfloor #1 \right \rfloor }
\providecommand{\qbm}[1]{\begin{bmatrix} #1 \end{bmatrix}}
\providecommand{\qpm}[1]{\begin{pmatrix} #1 \end{pmatrix}}
\providecommand{\norm}[1]{\left\lVert #1 \right\rVert}
\providecommand{\len}[1]{\left| #1 \right|}
\newcommand{\cmark}{\ding{51}}%
\newcommand{\xmark}{\ding{55}}%

\begin{document}



\title{\textbf{MATH 307: Individual Homework 11}}


\author{Shaochen (Henry) ZHONG, \ilc{sxz517@case.edu}}

\date{Due and submitted on 03/15/2021 \\ Spring 2021, Dr. Guo}
\maketitle



\subsection*{Problem 1}
\textit{See HW instruction.}\newline

It is a linear mapping as it satisfies the following conditions (for $\lambda \in \mathbb{R}$):

\begin{align*}
    f(\lambda u) &= f(\lambda (a_0 + a_1 x + a_2 x^2 + a_3 x^3) \\
    &= f(\lambda a_0 + \lambda a_1 x + \lambda a_2 x^2 + \lambda a_3 x^3) \\
    &= \lambda a_3 = \lambda f(u)\\
    f(u + v) &= f(a_0 + a_1 x + a_2 x^2 + a_3 x^3 + b_0 + b_1 x + b_2 x^2 + b_3 x^3) \\
    &= f( (a_0 + b_0) + (a_1 + b_1) x + (a_2 + b_2) x^2 + (a_3 + b_3) x^3 \\
    &= a_3 + b_3 = f(u) + f(v)
\end{align*}




\subsection*{Problem 2}
\textit{See HW instruction.}\newline

\begin{itemize}
    \item $A$ is diagonal, and therefore also upper-/lower-triangular, and symmatric. Since it is a matrix with elements in $\mathbb{R}$, it is also hermitian.
    \item $B$ is not diagonal nor upper-triangular, but it is lower triangular. It is neither symmetric nor hermitian.
    \item $C$ is not diagonal not lower-triangular, but it is upper-triangular. It is neither symmetric nor hermitian.
    \item $D$ is not diagonal, not upper- or lower-triangular. It is symmatric but not hermitian as for row two in $D$, it should be equal to $[2, 4, i]$ but it is not.
    \item $E$ is not diagonal, not upper- or lower-triangular. It is both symmatric and hermitian.
    \item $F$ is not a square matrix so it is not diagonal, not upper-, and not lower-triangular. For the same reason it is also not symmatric nor hermitian.
\end{itemize}

\noindent For easy grading:
\begin{table}
    \centering
    \begin{tabular}{ c | c c c c c c c c c}
        \hline
        matrix & Diagonal & UT & LT & Symmatric & Hermitian \\
        \hline
        $A$ & \cmark & \cmark & \cmark & \cmark & \cmark \\
        $B$ & \xmark & \xmark & \cmark & \xmark & \xmark \\
        $C$ & \xmark & \cmark & \xmark & \xmark & \xmark \\
        $D$ & \xmark & \xmark & \xmark & \cmark & \xmark \\
        $E$ & \xmark & \xmark & \xmark & \cmark & \cmark \\
        $F$ & \xmark & \xmark & \xmark & \xmark & \xmark \\
    \end{tabular}
\end{table}


\subsection*{Problem 3}
\textit{See HW instruction.}\newline


First start with LHS, known that the $ij$-th entry in $(\alpha A + \beta B)$ is $\alpha A_{ij} + \beta B_{ij}$. Thus, by the definition of transpose, the $ij$-th entry in $(\alpha A + \beta B)^T$ must be $\alpha A_{ji} + \beta B_{ji}$.

Now inspect the RHS, we have the $ij$-th entry in $\alpha A^T$ to be $\alpha A_{ji}$; similiarily, the $ij$-th entry in $\beta B^T$ is $\beta B_{ji}$. So the RHS equals to $\alpha A_{ji} + \beta B_{ji}$.\newline

As the two sides are equal, the equality-in-question is therefore proven.

\subsection*{Problem 4}
\textit{See HW instruction.}\newline

Being the diagonal entries of a Hermitian matrix $A$, each entry has to be real valued as we must have $A_{ij} = \overline{A_{ji}}$. This is only possible when the imaginary part of $A_{ij}$ is $0i$, so it must be a real valued number.

\end{document}

