\documentclass[11pt]{article}
\usepackage{setspace}
\setstretch{1}
\usepackage{amsmath,amssymb, amsthm}
\usepackage{graphicx}
\usepackage{bm}
\usepackage[hang, flushmargin]{footmisc}
\usepackage[colorlinks=true]{hyperref}
\usepackage[nameinlink]{cleveref}
\usepackage{footnotebackref}
\usepackage{url}
\usepackage{listings}
\usepackage[most]{tcolorbox}
\usepackage{inconsolata}
\usepackage[papersize={8.5in,11in}, margin=1in]{geometry}
\usepackage{float}
\usepackage{caption}
\usepackage{esint}
\usepackage{url}
\usepackage{enumitem}
\usepackage{subfig}
\usepackage{wasysym}
\newcommand{\ilc}{\texttt}
\usepackage{etoolbox}
\usepackage{algorithm}
\usepackage{changepage}
% \usepackage{algorithmic}
\usepackage[noend]{algpseudocode}
\usepackage{tikz}
\usepackage{pifont}
\usepackage{gensymb}
\usetikzlibrary{matrix,positioning,arrows.meta,arrows}
\patchcmd{\thebibliography}{\section*{\refname}}{}{}{}
% \PassOptionsToPackage{hyphens}{url}\usepackage{hyperref}

\providecommand{\myceil}[1]{\left \lceil #1 \right \rceil }
\providecommand{\myfloor}[1]{\left \lfloor #1 \right \rfloor }
\providecommand{\qbm}[1]{\begin{bmatrix} #1 \end{bmatrix}}
\providecommand{\qpm}[1]{\begin{pmatrix} #1 \end{pmatrix}}
\providecommand{\norm}[1]{\left\lVert #1 \right\rVert}
\providecommand{\len}[1]{\left| #1 \right|}
\newcommand{\cmark}{\ding{51}}%
\newcommand{\xmark}{\ding{55}}%

\begin{document}



\title{\textbf{MATH 307: Individual Homework 13}}


\author{Shaochen (Henry) ZHONG, \ilc{sxz517@case.edu}}

\date{Due and submitted on 03/26/2021 \\ Spring 2021, Dr. Guo}
\maketitle



\subsection*{Problem 1}
\textit{See HW instruction.}\newline

Say we have the angle between $Qx$ and $Qy$ being $\alpha$ and the angle between $x$ and $y$ being $\beta$, we have:

\begin{align*}
    \cos \beta &= \frac{<x, y>}{\norm{x} \norm{y}} \\
    &=\frac{<x, y>}{\sqrt{<x, x>} \sqrt{<y, y>}} \\
    \cos \alpha &= \frac{<Qx, Qy>}{\norm{Qx} \norm{Qy}} \\
    &=\frac{<Qx, y>}{\sqrt{<Qx, Qx>} \sqrt{<Qy, Qy>}} = \frac{(Qy)^* Qx}{\sqrt{(Qx)^* Qx}\sqrt{(Qy)^* Qy}} \\
    &= \frac{y^* Q^* Qx}{\sqrt{(Qx)^* Qx}\sqrt{(Qy)^* Qy}} = \frac{y^* x}{\sqrt{(Qx)^* Qx}\sqrt{(Qy)^* Qy}} \\
    &= \frac{<x, y>}{\sqrt{<x, x>} \sqrt{<y, y>}} = \cos \beta
\end{align*}

Thus, the unitary matrix does not affect the angle vetween two vectors.

\subsection*{Problem 2}
\textit{See HW instruction.}\newline

For $N(A) = \{x \in F^n \mid Ax = 0 \}$, we have:

\begin{itemize}
    \item $0 \in N(A)$, as $A0 = 0$.
    \item If $x, y \in N(A)$, we also have $x + y \in N(A)$ as $A(x + y) = Ax + Ay = 0 + 0 = 0$.
    \item If $x \in N(A)$ and $c \in F$, we have $cx \in N(A)$ as $A(cx) = cAx = c \cdot 0 = 0$.
\end{itemize}

Thus, $N(A)$ is a subspace of $F^n$.

\subsection*{Problem 3}
\textit{See HW instruction.}\newline

Conduct $R2 - R1$, we have $rref(A) = \qbm{1 & 2 \\ 0 & 0}$ and we know that the pivot column is the first column. Therefore, we have $\qbm{1 \\ 1}$ to be the basis for $A$.

\subsection*{Problem 4}
\textit{See HW instruction.}\newline

For $v \in range(AB)$, we must have $ABx = v$ for any $v$, where $ABx = A(Bx) = v$; so such $v$ will also be in $\in range(A)$.

\end{document}

