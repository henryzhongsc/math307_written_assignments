\documentclass[11pt]{article}
\usepackage{setspace}
\setstretch{1}
\usepackage{amsmath,amssymb, amsthm}
\usepackage{graphicx}
\usepackage{bm}
\usepackage[hang, flushmargin]{footmisc}
\usepackage[colorlinks=true]{hyperref}
\usepackage[nameinlink]{cleveref}
\usepackage{footnotebackref}
\usepackage{url}
\usepackage{listings}
\usepackage[most]{tcolorbox}
\usepackage{inconsolata}
\usepackage[papersize={8.5in,11in}, margin=1in]{geometry}
\usepackage{float}
\usepackage{caption}
\usepackage{esint}
\usepackage{url}
\usepackage{enumitem}
\usepackage{subfig}
\usepackage{wasysym}
\newcommand{\ilc}{\texttt}
\usepackage{etoolbox}
\usepackage{algorithm}
\usepackage{changepage}
% \usepackage{algorithmic}
\usepackage[noend]{algpseudocode}
\usepackage{tikz}
\usepackage{pifont}
\usepackage{gensymb}
\usetikzlibrary{matrix,positioning,arrows.meta,arrows}
\patchcmd{\thebibliography}{\section*{\refname}}{}{}{}
% \PassOptionsToPackage{hyphens}{url}\usepackage{hyperref}

\providecommand{\myceil}[1]{\left \lceil #1 \right \rceil }
\providecommand{\myfloor}[1]{\left \lfloor #1 \right \rfloor }
\providecommand{\qbm}[1]{\begin{bmatrix} #1 \end{bmatrix}}
\providecommand{\qpm}[1]{\begin{pmatrix} #1 \end{pmatrix}}
\providecommand{\norm}[1]{\left\lVert #1 \right\rVert}
\providecommand{\len}[1]{\left| #1 \right|}
\newcommand{\cmark}{\ding{51}}%
\newcommand{\xmark}{\ding{55}}%

\begin{document}



\title{\textbf{MATH 307: Individual Homework 15}}


\author{Shaochen (Henry) ZHONG, \ilc{sxz517@case.edu}}

\date{Due and submitted on 03/31/2021 \\ Spring 2021, Dr. Guo}
\maketitle



\subsection*{Problem 1}
\textit{Textbook page 80, problem 1.}\newline

\textbf{W.T.S.} $\norm{Ax} \leq \norm{A} \norm{x}$

\begin{align*}
    \text{Since} \ \norm{A} &= \text{sup}_{x \neq 0} \frac{\norm{Ax}}{\norm{x}} \\
    \text{There must be } \ \frac{\norm{Ax}}{\norm{x}} &\leq \norm{A} \\
    \Longrightarrow \norm{Ax} &\leq \norm{A}\norm{x}
\end{align*}

\textbf{W.T.S.} $\norm{AB} \leq \norm{A} \norm{B}$

\begin{align*}
    \norm{AB} &= \text{sup}_{x \neq 0} \frac{\norm{(AB)x}}{\norm{x}}= \text{sup}_{\norm{x} = 1}  \norm{A(Bx)} \\
    &\leq \text{sup}_{\norm{x} = 1} \norm{A} \norm{B(x)} \\
    &\leq \text{sup}_{\norm{x} = 1} \norm{A} \norm{B} \\ \norm{x} &= \norm{A} \norm{B} \\
    \norm{AB} &\leq \norm{A} \norm{B}
\end{align*}



\subsection*{Problem 2}
\textit{Textbook page 80, problem 2.}\newline

\paragraph*{$\infty$ norm}

As $\norm{A}_\infty = \max_{i} \sum\limits_{j = 1}^{n} \len{a_{ij}}$, we will take the maximum row sum of the matrix, which is $\len{-20} + \len{20} + \len{2} + \len{-2} + \len{0} = 44$.

\paragraph*{1-norm}

Similar to the $\infty$ norm above, as $\norm{A}_1 = \max_{j} \sum\limits_{i = 1}^{m} \len{a_{ij}}$, we will take the maximum column sum of the matrix, which is $\len{4} + \len{-1} + \len{5} + \len{20} = 30$.

\paragraph*{Frobenius norm}

As $\norm{A}_F = \sqrt{\sum\limits_i^m \sum\limits_j^n \len{a_{ij}}^2}$, we will take the square of each entries, add them together, and square root it. For simplicity I will omit the copying, the result should be $\sqrt{1279}$.


\subsection*{Problem 3}
\textit{Textbook page 80, problem 3.}\newline

\textbf{W.T.S.} $\norm{PA}_2 = \norm{A}_2$

Known that for an orthogonal matrix $A$, there must be $A^T A = I$; and also known $\norm{A} = \sqrt{A^T A}$. Note

\begin{align*}
    \norm{QA} &= \sqrt{(QA)^T QA} = \sqrt{A^T Q^T Q A} \\
    &= \sqrt{A^T I A} = \sqrt{A^T A} \\
    &= \norm{A}
\end{align*}

Since this relationship is not restricted to any $p$-norm, we may have $\norm{QA}_2 = \norm{A}_2$.\newline

\textbf{W.T.S.} $\norm{AQ}_2 = \norm{A}_2$

Known that $\norm{Qx}_2 = \sqrt{<Qx, Qx>} = \sqrt{<x, Q^T Q x} = \sqrt{<x, x>} = \norm{x}_2$, we have:

\begin{align*}
    \norm{AQ}_2 &= \text{sup}_{\norm{x}_2 = 1} \norm{AQx}_2 =  \text{sup}_{\norm{Qx}_2 = 1} \norm{A(Qx)}_2 \\
    &= \text{sup}_{\norm{y}_2 = 1} \norm{Ay}_2 = \norm{A}_2
\end{align*}

\subsection*{Problem 4}
\textit{Textbook page 80, problem 4.}\newline

$\max(A) = \max \{a_{ij}\}$ is not a norm as it does not support \textit{non-negetivity} $\max(A) > 0$ when $A \neq 0$. Take a matrix with all negative entries, say $A = \qbm{-1 & -1 \\ -1 & -1}$. We have $\max(A) = -1 < 0$, non-negetivity is violated and therefore not a norm.\newline

$\max(\len{AB}) = \max \{|a_{ij}|\}$ is also not a norm as it does not support \textit{submultiplicative} $\max(AB) \leq \max(A)\max(B)$. Take $A = B = \qbm{1 & 1 \\ 1 & 1}$, we have:

\begin{align*}
    \max(\len{AB}) &= \max(\qbm{1 & 1 \\ 1 & 1}\qbm{1 & 1 \\ 1 & 1}) = \max(\qbm{\len{2} & \len{2} \\ \len{2} & \len{2}}) = 2 \\
    \max(\len{A})\max(\len{B}) &= \max(\qbm{\len{1} & \len{1} \\ \len{1} & \len{1}}) \max(\qbm{\len{1} & \len{1} \\\len{1} & \len{1}}) = 1 \\
    \Longrightarrow \max(AB) &> \max(A)\max(B)
\end{align*}

As submultiplicative is violated, it is also not a norm.




\end{document}

