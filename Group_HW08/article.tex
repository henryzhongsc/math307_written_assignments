\documentclass[11pt]{article}
\usepackage{setspace}
\setstretch{1}
\usepackage{amsmath,amssymb, amsthm}
\usepackage{graphicx}
\usepackage{bm}
\usepackage[hang, flushmargin]{footmisc}
\usepackage[colorlinks=true]{hyperref}
\usepackage[nameinlink]{cleveref}
\usepackage{footnotebackref}
\usepackage{url}
\usepackage{listings}
\usepackage[most]{tcolorbox}
\usepackage{inconsolata}
\usepackage[papersize={8.5in,11in}, margin=1in]{geometry}
\usepackage{float}
\usepackage{caption}
\usepackage{esint}
\usepackage{url}
\usepackage{enumitem}
\usepackage{subfig}
\usepackage{wasysym}
\newcommand{\ilc}{\texttt}
\usepackage{etoolbox}
\usepackage{algorithm}
\usepackage{changepage}
% \usepackage{algorithmic}
\usepackage[noend]{algpseudocode}
\usepackage{tikz}
\usepackage{gensymb}
\usetikzlibrary{matrix,positioning,arrows.meta,arrows}
\patchcmd{\thebibliography}{\section*{\refname}}{}{}{}
% \PassOptionsToPackage{hyphens}{url}\usepackage{hyperref}

\providecommand{\myceil}[1]{\left \lceil #1 \right \rceil }
\providecommand{\myfloor}[1]{\left \lfloor #1 \right \rfloor }
\providecommand{\qbm}[1]{\begin{bmatrix} #1 \end{bmatrix}}
\providecommand{\qpm}[1]{\begin{pmatrix} #1 \end{pmatrix}}
\providecommand{\norm}[1]{\left\lVert #1 \right\rVert}
\providecommand{\len}[1]{\left| #1 \right|}

\begin{document}




\title{\textbf{MATH 307: Group Homework 8}}


\author{\textit{Group 8}\\
Shaochen (Henry) ZHONG, Zhitao (Robert) CHEN, John MAYS, Huaijin XIN\\ \ilc{\{sxz517, zxc325, jkm100, hxx200\}@case.edu}}

\date{Due and submitted on 04/09/2021 \\ Spring 2021, Dr. Guo}
\maketitle




\subsection*{Problem 1}
\textit{See HW instruction.}\newline

\begin{align*}
    \det(A - \lambda I) &= \det \qbm{3- \lambda & 0 & 1 \\ 0 & 2-\lambda & 1 \\ 0 & 0 & 2-\lambda} \\
    &= -\lambda^3 + 7 \lambda ^2 - 16\lambda + 12 = -(\lambda - 2)^2 (\lambda - 3) = 0 \\
    \Longrightarrow& \begin{cases}
        \lambda_1 = 2 \\
        \lambda_2 = 3
    \end{cases}
\end{align*}

\begin{align*}
    0 &= (A - \lambda_1 I) v_{\lambda_1} \\
    &= \qbm{3- \lambda_1 & 0 & 1 \\ 0 & 2-\lambda_1 & 1 \\ 0 & 0 & 2-\lambda_1} v \\
    &= \qbm{3- 2 & 0 & 1 \\ 0 & 2-2 & 1 \\ 0 & 0 & 2-2} v \Rightarrow \begin{cases}
        v_1 + v_3 = 0 \\
        0v_2 + v_3 = 0 \\
        0v_3 = 0
    \end{cases} \\
    \Longrightarrow v_{\lambda_1} &= \qbm{0 \\ 1 \\ 0}
\end{align*}

Known that $\lambda_1 = 2$ has an algebraic multiplicity of 2 (appered twice as the root); but the eigenspace of $\lambda_1$ is spanned by $ \qbm{0 \\ 1 \\ 0}$, which means it only has a geometric multiplicity of 1. Thus, the matrix is defective.

\subsection*{Problem 2}
\textit{See HW instruction.}\newline
% https://math.stackexchange.com/questions/762984/whats-the-proof-stategy-for-hermitian-matrix-has-orthogonal-eigenvectors-for-d

Known that $A = A^*$, let $Av = \lambda v$, $Aw = \mu w$ and we must have:

\begin{align*}
    Aw &= \mu w \\
    v^* Aw &= v^* \mu w = \mu (v^* w) \\
    Aw &= A^*w \\
    v^* A^* w  &= (Av)^* w = (\lambda v)^* w = \lambda (v^* w) \\
    \Longrightarrow \mu (v^* w) &= \lambda (v^* w)
\end{align*}

Since $\lambda, \mu \neq 0$ and $\lambda \neq \mu$ for being distinct eigenvalues, there must be $v^*w = 0 = <w, v>$. Thus, the eigenvectors-in-question are orthogonal.


\subsection*{Problem 3}
\textit{See HW instruction.}\newline

$A\in F^{n \times n} = U\Sigma V^{*}$ where $U, V$ are unitary and $\Sigma$ is diagonal, we must have:

\begin{align*}
    A^{*}A &=(U \Sigma V^{*})^{*}(U \Sigma V^{*}) \\
    &=(V \Sigma^{*} U^{*})(U \Sigma V^{*}) \\
    &= V \Sigma^{*} U^{*}U \Sigma V^{*} \\
    &= V \Sigma^{*} I \Sigma V^{*} \\
    &= V \Sigma^{*} \Sigma V^{*} \\
    &= V (\Sigma^{*} \Sigma) V^{*}
\end{align*}

Since $V$ is a unitary matrix, it is also invertible, and $V^{*}=V^{-1}$. Therefore:

\begin{equation*}
    V (\Sigma^{*} \Sigma) V^{*} = V (\Sigma^{*} \Sigma) V^{-1}
\end{equation*}

Furthermore, the resultant matrix of $\Sigma^{*}\Sigma$ will also be a diagonal matrix $=\text{diag}(|\sigma_1|^2,|\sigma_2|^2,\dots, |\sigma_n|^2)$.\newline

Since $V$ is invertible, $V^{*} = V^{-1}$, and $\Sigma^{*}\Sigma$ is a diagonal matrix, $V (\Sigma^{*}\Sigma) V^{*}$ is an eigendecomposition of $A^{*}A$.

\subsection*{Problem 4}
\textit{See HW instruction.}\newline

\begin{align*}
    Ax &= b \\
    U \Sigma V^T x &= b \\
    (V \Sigma^{-1} U^T) U \Sigma V^T x &= (V \Sigma^{-1} U^T)b \\
    V \Sigma^{-1} (U^T U) \Sigma V^T x &= V \Sigma^{-1} U^T b \\
    V (\Sigma^{-1} \Sigma) V^T x &= V \Sigma^{-1} U^T b \\
    V V^T x &=V \Sigma^{-1} U^T b \\
    x &= V \Sigma^{-1} U^T b
\end{align*}

Subsitute the known values into above equation, we have:

\begin{align*}
    x &= V \Sigma^{-1} U^T b \\
    &= \qbm{0 & 1 \\ 1 & 0} \qbm{2 & 0 \\ 0 & 1}^{-1} \qbm{\frac{\sqrt{2}}{2} & -\frac{\sqrt{2}}{2} \\ \frac{\sqrt{2}}{2} & \frac{\sqrt{2}}{2}}^T \qbm{-\frac{3}{2} \sqrt{2} \\ -\frac{\sqrt{2}}{2}} \\
    &= \qbm{0 & 1 \\ 1 & 0} \qbm{\frac{1}{2} & 0 \\ 0 & 1} \qbm{\frac{\sqrt{2}}{2} & \frac{\sqrt{2}}{2} \\ -\frac{\sqrt{2}}{2} & \frac{\sqrt{2}}{2}} \qbm{-\frac{3}{2} \sqrt{2} \\ -\frac{\sqrt{2}}{2}} \\
    &= \qbm{1 \\ -1}
\end{align*}

Thus, $x = \qbm{1 \\ -1}$.

% {{0, 1}, {1, 0}} {{1/2, 0}, {0, 1}} {{\sqrt{2}/2, \sqrt{2}/2}, {-\sqrt{2}/2, \sqrt{2}/2}} {{-3/2 \sqrt{2}},{-\sqrt{2}/2}}

\end{document}

