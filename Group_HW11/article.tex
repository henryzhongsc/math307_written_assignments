\documentclass[11pt]{article}
\usepackage{setspace}
\setstretch{1}
\usepackage{amsmath,amssymb, amsthm}
\usepackage{graphicx}
\usepackage{bm}
\usepackage[hang, flushmargin]{footmisc}
\usepackage[colorlinks=true]{hyperref}
\usepackage[nameinlink]{cleveref}
\usepackage{footnotebackref}
\usepackage{url}
\usepackage{listings}
\usepackage[most]{tcolorbox}
\usepackage{inconsolata}
\usepackage[papersize={8.5in,11in}, margin=1in]{geometry}
\usepackage{float}
\usepackage{caption}
\usepackage{esint}
\usepackage{url}
\usepackage{enumitem}
\usepackage{subfig}
\usepackage{wasysym}
\newcommand{\ilc}{\texttt}
\usepackage{etoolbox}
\usepackage{algorithm}
\usepackage{changepage}
% \usepackage{algorithmic}
\usepackage[noend]{algpseudocode}
\usepackage{tikz}
\usepackage{gensymb}
\usetikzlibrary{matrix,positioning,arrows.meta,arrows}
\patchcmd{\thebibliography}{\section*{\refname}}{}{}{}
% \PassOptionsToPackage{hyphens}{url}\usepackage{hyperref}

\providecommand{\myceil}[1]{\left \lceil #1 \right \rceil }
\providecommand{\myfloor}[1]{\left \lfloor #1 \right \rfloor }
\providecommand{\qbm}[1]{\begin{bmatrix} #1 \end{bmatrix}}
\providecommand{\qpm}[1]{\begin{pmatrix} #1 \end{pmatrix}}
\providecommand{\norm}[1]{\left\lVert #1 \right\rVert}
\providecommand{\len}[1]{\left| #1 \right|}

\begin{document}




\title{\textbf{MATH 307: Group Homework 11}}


\author{\textit{Group 8}\\
Shaochen (Henry) ZHONG, Zhitao (Robert) CHEN, John MAYS, Huaijin XIN\\ \ilc{\{sxz517, zxc325, jkm100, hxx200\}@case.edu}}

\date{Due and submitted on 04/30/2021 \\ Spring 2021, Dr. Guo}
\maketitle



\subsection*{Problem 1}
\textit{See HW instruction.}\newline

For $A, B, C, D$ we only have one entry on the first row bing non-zero, so we can only calculate the part leading by the non-zero entry.

\begin{align*}
    \det(A) &= - 1(1 \cdot 1) = -1 \\
    \det(B) &= 1(3 \cdot 1) = 3 \\
    \det(C) &= 1(1 \cdot 1) = 1 \\
    \det(D) &= 2(-5 \cdot 3) = -30
\end{align*}

For $D$, we have:
\begin{equation*}
    \det(D) = 1(1 \cdot 1) - 4(-1 \cdot 1) - 1(-1 \cdot 2) = 1 + 4 + 2 = 7
\end{equation*}

\subsection*{Problem 2}
\textit{See HW instruction.}\newline

Known that $A A^{-1} = I$, so there must be $\det(A A^{-1}) = \det(I) = 1$. Know that  determinant of the product is just product of determinants, we have $\det(A) \det(A^{-1}) = 1 \Longrightarrow \det(A^{-1}) = \frac{1}{\det(A)}$.

\subsection*{Problem 3}
\textit{See HW instruction.}\newline

Known that $\det(A) = \det(A^T)$ assuming $\det(A) \neq 0$ (\href{https://www2.math.upenn.edu/~ekorman/teaching/det.pdf}{ref.}), we have:

\begin{align*}
    \det(A^{-1} (B^T)^2) &= \det(A^{-1}) \det((B^T)^2) = \det(A^{-1}) \det(B^T)^2 = \frac{1}{\det(A)} \det(B)^2 = \frac{1}{2} (-1)^2 = \frac{1}{2} \\
    \det((B^T)^{-1} A^3) &= \det((B^T)^{-1}) \det(A)^3 =  \det(B^{-1})\det(A)^3 = \frac{1}{\det(B)} \det(A)^3 = -1 \cdot (2)^3 = -8
\end{align*}

\subsection*{Problem 4}
\textit{See HW instruction.}\newline

For $QQ^T = I$ there must be $\det(QQ^T) = \det(I) = 1$, so we have:

\begin{align*}
    1 &= \det(QQ^T) = \det(Q^T)\det(Q) = \det(Q) \det(Q)\\
    &= \det(Q)^2 \\
    \Longrightarrow \det(Q) &= \pm 1
\end{align*}

Therefore the determinant of an orthonormal basis $Q$ must be either 1 or $-1$.

\end{document}

