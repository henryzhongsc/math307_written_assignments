\documentclass[11pt]{article}
\usepackage{setspace}
\setstretch{1}
\usepackage{amsmath,amssymb, amsthm}
\usepackage{graphicx}
\usepackage{bm}
\usepackage[hang, flushmargin]{footmisc}
\usepackage[colorlinks=true]{hyperref}
\usepackage[nameinlink]{cleveref}
\usepackage{footnotebackref}
\usepackage{url}
\usepackage{listings}
\usepackage[most]{tcolorbox}
\usepackage{inconsolata}
\usepackage[papersize={8.5in,11in}, margin=1in]{geometry}
\usepackage{float}
\usepackage{caption}
\usepackage{esint}
\usepackage{url}
\usepackage{enumitem}
\usepackage{subfig}
\usepackage{wasysym}
\newcommand{\ilc}{\texttt}
\usepackage{etoolbox}
\usepackage{algorithm}
\usepackage{changepage}
% \usepackage{algorithmic}
\usepackage[noend]{algpseudocode}
\usepackage{tikz}
\usepackage{gensymb}
\usetikzlibrary{matrix,positioning,arrows.meta,arrows}
\patchcmd{\thebibliography}{\section*{\refname}}{}{}{}
% \PassOptionsToPackage{hyphens}{url}\usepackage{hyperref}

\providecommand{\myceil}[1]{\left \lceil #1 \right \rceil }
\providecommand{\myfloor}[1]{\left \lfloor #1 \right \rfloor }
\providecommand{\qbm}[1]{\begin{bmatrix} #1 \end{bmatrix}}
\providecommand{\qpm}[1]{\begin{pmatrix} #1 \end{pmatrix}}
\providecommand{\norm}[1]{\left\lVert #1 \right\rVert}
\providecommand{\len}[1]{\left| #1 \right|}

\begin{document}




\title{\textbf{MATH 307: Group Homework 6}}


\author{\textit{Group 8}\\
Shaochen (Henry) ZHONG, Zhitao (Robert) CHEN, John MAYS, Huaijin XIN\\ \ilc{\{sxz517, zxc325, jkm100, hxx200\}@case.edu}}

\date{Due and submitted on 03/19/2021 \\ Spring 2021, Dr. Guo}
\maketitle




\subsection*{Problem 1}
\textit{See HW instruction.}\newline


Assume the original vector $\qbm{x_0 \\ y_0}$ with a length of $r$ has a degree of $\theta$, we first reflect it about y-axis by swapping the $x$ value with $-x$ and make $A \qbm{x_0 \\ y_0} = \qbm{-x_0 \\ y_0}$, this means $A = \qbm{-1 & 0 \\ 0 & 1}$.


Say the $\qbm{-x_0 \\ y_0}$, we call it $\qbm{x_1 \\ y_1}$, got a degree of $\theta$, we then rotate it $\phi$ (in this case $\phi = \frac{\pi}{4}$) degrees more couterclock wisely to have $\qbm{x_2 \\ y_2}$:

\begin{align*}
    x_2 &= r \cos(\theta + \phi) = r(\cos \theta \cos \phi - \sin \theta \sin \phi) \\
    &= r \cos \theta \cos \phi - r \sin \theta \sin \phi \\
    &= x_1 \cos \phi - y_1 \sin \phi \\
    y_2 &= r \sin(\theta + \phi) = r(\sin \theta \cos \phi + \cos \theta \sin \phi) \\
    &= r\sin \theta \cos \phi + r\cos \theta \sin \phi \\
    &= y_1 \cos \phi + x_1 \sin \phi
\end{align*}

As we need $B \qbm{x_1 \\ y_1} = \qbm{x_2 \\ y_2}$, we must have:

\begin{align*}
    x_2 &= x_1 \cos \phi - y_1 \sin \phi \\
    y_2 &= y_1 \cos \phi + x_1 \sin \phi \\
    \qbm{x_2 \\ y_2} &= \qbm{\cos \phi & -\sin \phi \\ \sin \phi & \cos \phi} \cdot \qbm{x_1 \\ y_1} \\
    &\text{Since}\ \phi = \frac{\pi}{4} \\
    \qbm{x_2 \\ y_2} &= \qbm{\frac{1}{\sqrt{2}} & -\frac{1}{\sqrt{2}}  \\ \frac{1}{\sqrt{2}}  & \frac{1}{\sqrt{2}} } \cdot \qbm{x_1 \\ y_1} \\
    &= \qbm{\frac{1}{\sqrt{2}} & -\frac{1}{\sqrt{2}}  \\ \frac{1}{\sqrt{2}}  & \frac{1}{\sqrt{2}} } \cdot (\qbm{-1 & 0 \\ 0 & 1} \qbm{x_0 \\ y_0}) \\
    &= (\qbm{\frac{1}{\sqrt{2}} & -\frac{1}{\sqrt{2}}  \\ \frac{1}{\sqrt{2}}  & \frac{1}{\sqrt{2}} } \cdot \qbm{-1 & 0 \\ 0 & 1}) \qbm{x_0 \\ y_0} \\
    &= \qbm{-\frac{1}{\sqrt{2}} & -\frac{1}{\sqrt{2}} \\ -\frac{1}{\sqrt{2}} & \frac{1}{\sqrt{2}}} \cdot \qbm{x_0 \\ y_0}
\end{align*}

Thus, $C = \qbm{-\frac{1}{\sqrt{2}} & -\frac{1}{\sqrt{2}} \\ -\frac{1}{\sqrt{2}} & \frac{1}{\sqrt{2}}}$, for $C \cdot \qbm{x \\ y}$ will conduct the proposed operations on $\qbm{x \\ y}$.

\subsection*{Problem 2}
\textit{See HW instruction.}\newline

\paragraph*{LHS:}
The $ij$-th entry of $(A+B) $ is $a_{ij}+b_{ij}$\\
%The $ij$-th of $(A+B)^2$ can be thought of with the matrix-multiplication product of $(A+B)(A+B)$:\\
The $ij$-th entry of $(A+B)^2 = (A+B)(A+B)$ is $\sum_{k=1}^n (a_{ik}+b_{ik})(a_{kj}+b_{kj})= \sum_{k=1}^n a_{ik}a_{kj}+a_{ik}b_{kj}+b_{ik}a_{kj}+b_{ik}b_{kj}$
\paragraph*{RHS:}
The $ij$-th entry of $A^2$ is $\sum_{k=1}^n a_{ik}a_{kj}$\\
It follows that the $ij$-th entry of $B^2$ is $\sum_{k=1}^n b_{ik}b_{kj}$\\
The $ij$-th entry of $AB$ is $\sum_{k=1}^n a_{ik}b_{kj}$\\
Therefore the $ij$-th entry of $A^2+2AB+B^2$ is $\sum_{k=1}^n a_{ik}a_{kj} + 2a_{ik}b_{kj}+ b_{ik}b_{kj}$
\paragraph*{Conclusion:}
The respective $ij$-th entry of the LHS and the RHS of the equation are not equivalent unless $AB = BA$, which is not guaranteed; therefore $(A+B)^2=A^2+2AB+B^2$ is not true for two square matrices, $A$ and $B,$ of the same size.


\subsection*{Problem 3}
\textit{See HW instruction.}\newline

Known that $M_{ij}^* = \overline{M_{ji}}$, we have:

\begin{align*}
    LHS &= (AB)^*_{ij} = \overline{(AB)_{ji}} = \sum^{n}_{k} \overline{A_{jk} B_{ki}} \\
    RHS &= (B^* A^*)_{ij} = \sum^{n}_{k} B^{*}_{ik} A^{*}_{kj} = \sum^{n}_{k} \overline{B_{ki} A_{jk}} \\
    \Longrightarrow& \  (AB)^*_{ij} = (B^* A^*)_{ij}
\end{align*}

\subsection*{Problem 4}
\textit{See HW instruction.}\newline

Known that $(AB)_{ij} =\sum_{k=1}^{n}A_{ik}B_{kj}$. In the case that $i > j$, we have either $A_{ik} = 0$ (when $i > k$) or $B_{jk} = 0$ (when $k < j$). So the sum will always be zero for $(AB)_{ij}$ where $i > j$, and thus an upper-triangular matrix.

\subsection*{Problem 5}
\textit{See HW instruction.}\newline

$A \vec{x} = 0$ implies $\sum\limits_{i}^{n} \sum\limits_{k = 1}^{n} A_{ik} x_{k} = 0$, which can be rearranged to $\sum\limits_{k = 1}^{n} x_k v_k = 0$ for $v_i$ being the $i$-th column of $A$. If all the $v_i$ of $A$ are linearly independent, we must have $A \vec{x} = 0$ only when $\vec{x} = 0$\newline

\textbf{W.T.S.} $A^{-1}$ exists $\Longrightarrow A \vec{x} = 0$  only when $\vec{x} = 0$.

\begin{align*}
    A \vec{x} &= 0 \\
    A^{-1} A \vec{x} &= A^{-1} 0 \\
    I \vec{x} &= 0 \\
    \vec{x} = 0
\end{align*}

Thus, the first direction is therefore proven.\newline

\textbf{W.T.S.} $A \vec{x} = 0$ only when $\vec{x} = 0 \ \Longrightarrow A^{-1}$ exists.

Since all columns $x_i$ of $A$ are linearly independent, by definition any $\vec{b}$ within the defined vector space ($\mathbb{R}^n$ or $\mathbb{C}^n$) can be achived by a linear combination of columns of $A$, which implies there must be a solution for $A \vec{x} = \vec{b}$ for all legal $\vec{b}$.

Now we want to show that $AM = I$, since $I \in \mathbb{R}^n$ (or $I \in \mathbb{C}^n$, W.L.O.G.), there must be a solution $M$ for this equality. Thus, $A$ is invertable given all of its columns are linearly independent.\newline

As both directions are proven, the iff-relationship-in-question is therefore proven.


\end{document}

