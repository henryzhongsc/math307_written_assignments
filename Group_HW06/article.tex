\documentclass[11pt]{article}
\usepackage{setspace}
\setstretch{1}
\usepackage{amsmath,amssymb, amsthm}
\usepackage{graphicx}
\usepackage{bm}
\usepackage[hang, flushmargin]{footmisc}
\usepackage[colorlinks=true]{hyperref}
\usepackage[nameinlink]{cleveref}
\usepackage{footnotebackref}
\usepackage{url}
\usepackage{listings}
\usepackage[most]{tcolorbox}
\usepackage{inconsolata}
\usepackage[papersize={8.5in,11in}, margin=1in]{geometry}
\usepackage{float}
\usepackage{caption}
\usepackage{esint}
\usepackage{url}
\usepackage{enumitem}
\usepackage{subfig}
\usepackage{wasysym}
\newcommand{\ilc}{\texttt}
\usepackage{etoolbox}
\usepackage{algorithm}
\usepackage{changepage}
% \usepackage{algorithmic}
\usepackage[noend]{algpseudocode}
\usepackage{tikz}
\usepackage{gensymb}
\usetikzlibrary{matrix,positioning,arrows.meta,arrows}
\patchcmd{\thebibliography}{\section*{\refname}}{}{}{}
% \PassOptionsToPackage{hyphens}{url}\usepackage{hyperref}

\providecommand{\myceil}[1]{\left \lceil #1 \right \rceil }
\providecommand{\myfloor}[1]{\left \lfloor #1 \right \rfloor }
\providecommand{\qbm}[1]{\begin{bmatrix} #1 \end{bmatrix}}
\providecommand{\qpm}[1]{\begin{pmatrix} #1 \end{pmatrix}}
\providecommand{\norm}[1]{\left\lVert #1 \right\rVert}
\providecommand{\len}[1]{\left| #1 \right|}

\begin{document}




\title{\textbf{MATH 307: Group Homework 6}}


\author{\textit{Group 8}\\
Shaochen (Henry) ZHONG, Zhitao (Robert) CHEN, John MAYS, Huaijin XIN\\ \ilc{\{sxz517, zxc325, jkm100, hxx200\}@case.edu}}

\date{Due and submitted on 03/19/2021 \\ Spring 2021, Dr. Guo}
\maketitle




\subsection*{Problem 1}
\textit{See HW instruction.}\newline
\subsubsection*{Reflection: }
For a reflection across the y-axis, the x-component of a vector must become -x\\
$A\begin{pmatrix}x\\ y\end{pmatrix} = \begin{pmatrix}-x\\ y\end{pmatrix}$\\
If $A = \begin{pmatrix}-1 & 0\\ 0 & 1\end{pmatrix}$, then $A\begin{pmatrix}x\\ y\end{pmatrix}=\begin{pmatrix}-1(x)+0(x)\\ 0(y)+1(y)\end{pmatrix}=\begin{pmatrix}-x\\ y\end{pmatrix}$\\
Therefore $A$ is the reflection matrix.
\subsubsection*{Rotation: }
To obtain the rotation matrix, we can compute the default Cartesian rotation matrix with $\theta = \frac{\pi}{4}$\\
$B=\begin{pmatrix}\cos{\frac{\pi}{4}} & -\sin{\frac{\pi}{4}}\\ \sin{\frac{\pi}{4}} & \cos{\frac{\pi}{4}}\ \end{pmatrix}=\begin{pmatrix}\frac{1}{\sqrt{2}} & -\frac{1}{\sqrt{2}}\\ \frac{1}{\sqrt{2}} & \frac{1}{\sqrt{2}} \end{pmatrix}$
Therefore $B$ is the rotation matrix.
\subsubsection*{Entire Operation:}
The matrix for the entire operation is the reflection matrix times the rotation matrix,\\
$AB=\begin{pmatrix}-1 & 0\\ 0 & 1\end{pmatrix}\begin{pmatrix}\frac{1}{\sqrt{2}} & -\frac{1}{\sqrt{2}}\\ \frac{1}{\sqrt{2}} & \frac{1}{\sqrt{2}} \end{pmatrix}=\begin{pmatrix}(-1)(\frac{1}{\sqrt{2}}) +(0)(\frac{1}{\sqrt{2}})& (-1)(-\frac{1}{\sqrt{2}})+(0)(\frac{1}{\sqrt{2}})\\ (0)(\frac{1}{\sqrt{2}}) +(1)(\frac{1}{\sqrt{2}})& (0)(-\frac{1}{\sqrt{2}})+(1)(\frac{1}{\sqrt{2}}) \end{pmatrix}=\begin{pmatrix}-\frac{\sqrt{2}}{2} & -\frac{\sqrt{2}}{2}\\ \frac{\sqrt{2}}{2} & \frac{\sqrt{2}}{2} \end{pmatrix}$
\pagebreak

\subsection*{Problem 2}
\textit{See HW instruction.}\newline
quation in question: $(A+B)^2=A^2+2AB+B^2$ for two square matrices of the same size, $A$ and $B$.
\subsubsection*{LHS:}
The $ij$-th entry of $(A+B) $ is $a_{ij}+b_{ij}$\\
%The $ij$-th of $(A+B)^2$ can be thought of with the matrix-multiplication product of $(A+B)(A+B)$:\\
The $ij$-th entry of $(A+B)^2 = (A+B)(A+B)$ is $\sum_{k=1}^n (a_{ik}+b_{ik})(a_{kj}+b_{kj})= \sum_{k=1}^n a_{ik}a_{kj}+a_{ik}b_{kj}+b_{ik}a_{kj}+b_{ik}b_{kj}$
\subsubsection*{RHS:}
The $ij$-th entry of $A^2$ is $\sum_{k=1}^n a_{ik}a_{kj}$\\
It follows that the $ij$-th entry of $B^2$ is $\sum_{k=1}^n b_{ik}b_{kj}$\\
The $ij$-th entry of $AB$ is $\sum_{k=1}^n a_{ik}b_{kj}$\\
Therefore the $ij$-th entry of $A^2+2AB+B^2$ is $\sum_{k=1}^n a_{ik}a_{kj} + 2a_{ik}b_{kj}+ b_{ik}b_{kj}$
\subsubsection*{Conclusion:}
The respective $ij$-th entry of the LHS and the RHS of the equation are not equivalent, therefore $(A+B)^2=A^2+2AB+B^2$ is not true for two square matrices, $A$ and $B,$ of the same size.

\subsection*{Problem 3}
\textit{See HW instruction.}\newline

Known that $M_{ij}^* = \overline{M_{ji}}$, we have:

\begin{align*}
    LHS &= (AB)^*_{ij} = \overline{(AB)_{ji}} = \sum^{n}_{k} \overline{A_{jk} B_{ki}} \\
    RHS &= (B^* A^*)_{ij} = \sum^{n}_{k} B^{*}_{ik} A^{*}_{kj} = \sum^{n}_{k} \overline{B_{ki} A_{jk}} \\
    \Longrightarrow& \  (AB)^*_{ij} = (B^* A^*)_{ij}
\end{align*}

\subsection*{Problem 4}
\textit{See HW instruction.}\newline

\subsection*{Problem 5}
\textit{See HW instruction.}\newline


\end{document}

