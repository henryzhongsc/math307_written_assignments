\documentclass[11pt]{article}
\usepackage{setspace}
\setstretch{1}
\usepackage{amsmath,amssymb, amsthm}
\usepackage{graphicx}
\usepackage{bm}
\usepackage[hang, flushmargin]{footmisc}
\usepackage[colorlinks=true]{hyperref}
\usepackage[nameinlink]{cleveref}
\usepackage{footnotebackref}
\usepackage{url}
\usepackage{listings}
\usepackage[most]{tcolorbox}
\usepackage{inconsolata}
\usepackage[papersize={8.5in,11in}, margin=1in]{geometry}
\usepackage{float}
\usepackage{caption}
\usepackage{esint}
\usepackage{url}
\usepackage{enumitem}
\usepackage{subfig}
\usepackage{wasysym}
\newcommand{\ilc}{\texttt}
\usepackage{etoolbox}
\usepackage{algorithm}
\usepackage{changepage}
% \usepackage{algorithmic}
\usepackage[noend]{algpseudocode}
\usepackage{tikz}
\usepackage{pifont}
\usepackage{gensymb}
\usetikzlibrary{matrix,positioning,arrows.meta,arrows}
\patchcmd{\thebibliography}{\section*{\refname}}{}{}{}
% \PassOptionsToPackage{hyphens}{url}\usepackage{hyperref}

\providecommand{\myceil}[1]{\left \lceil #1 \right \rceil }
\providecommand{\myfloor}[1]{\left \lfloor #1 \right \rfloor }
\providecommand{\qbm}[1]{\begin{bmatrix} #1 \end{bmatrix}}
\providecommand{\qpm}[1]{\begin{pmatrix} #1 \end{pmatrix}}
\providecommand{\norm}[1]{\left\lVert #1 \right\rVert}
\providecommand{\len}[1]{\left| #1 \right|}
\newcommand{\cmark}{\ding{51}}%
\newcommand{\xmark}{\ding{55}}%

\begin{document}



\title{\textbf{MATH 307: Individual Homework 18}}


\author{Shaochen (Henry) ZHONG, \ilc{sxz517@case.edu}}

\date{Due and submitted on 04/14/2021 \\ Spring 2021, Dr. Guo}
\maketitle



\subsection*{Problem 1}
\textit{See HW instruction.}\newline

The question worded as \textit{``find a basis for both $range(A)$ and $range(A^*)$''}, I am a bit unsure if I should find one basis for $range(A)$ and another basis $range(A^*)$, or should I find a single basis for both of them -- it seems like the question is asking the latter one, but I don't think that is possible given that $A \in F^{m \times n}$; and if $m \neq n$, the dimension of row and column space is intrinctly different.\newline

Let $\sigma_1, \sigma_2, \dots, \sigma_r$ be the nonzero singular values in $\Sigma$ of $A = U \Sigma V^*$, we know that $\{u_1, u_2, ..., u_r\}$ is a basis for $range(A)$ because of the following.

\textbf{W.T.S.} $range(A) \subset span(u_1, u_2, ..., u_r)$

$\forall y \in range(A)$, $y = Ax$ for some $x \in F^n$. Since we know that $V$ is unitary, columns of $V$ form a basis for $F^n$. Thus, we have $x = \sum \alpha_i v_i$; also known that $AV = U \Sigma$, we have:

\begin{align*}
    y &= Ax = A(\sum_{i=1}^r \alpha_i v_i) \\
    &= \sum_{i=1}^r \alpha_i A v_i \\
    &= \sum_{i=1}^r\alpha_i \sigma_i u_i \in span(u_1, u_2, ..., u_r)
\end{align*}

This implies $range(A) \subset span(u_1, u_2, \dots, u_r)$\newline

\textbf{W.T.S.} $span(u_1, u_2, \dots, u_r) \subset  range(A)$

\begin{align*}
    AV &= U \Sigma \\
    A v_i &= \sigma_i u_i \ \text{for} \ i = 1, \dots, r\\
    u_i &= A(\frac{v_i}{\sigma_i}) \\
    \Longrightarrow& u_i \in range(A)
\end{align*}

This implies $span(u_1, u_2, \dots, u_r) \subset range(A)$. Combine both findings, we have $range(A) = span(u_1, u_2, \dots, u_r)$. As $u_1, u_2, \dots, u_r$ are linearly independent, it is a basis for $range(A)$ the column space of $A$.\newline

Similarily, we have $A^* = V \Sigma U^*$ for $V$ being unitary (implies orthogonal and linearly independent columns). So, we have $\{v_1, v_2, \dots, v_r\}$ to be basis for $range(A^*)$ the row space of $A$.\newline

Since both $\{u_1, u_2, \dots, u_r\}$ and $\{v_1, v_2, \dots, v_r\}$ have a dimension of $r$. The column rank is same as the row rank of $A$.



\subsection*{Problem 2}
\textit{See HW instruction.}\newline

\subsubsection*{(a)}
As we have previously established in the above \textbf{Problem 1}. The row rank of $A$ is the number of nonzero singular values in $\Sigma$ of $A = U \Sigma V^*$. In this case, we have 5 nonzero singular values, so the row rank of $A$ is 5. The orthonormal basis of $range(A^*)$ will therefore be $\{v_1, v_2, \dots, v_5 \}$.

For proving. We know that $\{v_1, v_2, \dots, v_5 \}$ is a orthonormal set as it came out from an unitary matrix $V$. We also know there must be $span(v_1, v_2, \dots, v_5) = range(A^*)$ as we have previously showned in \textbf{Problem 1}. Thus, $\{v_1, v_2, \dots, v_5 \}$ is a basis of $range(A^*)$.

\subsubsection*{(b)}

The rank-nullity theorem tells us that $n = r + l$ for $r$ stands for rank and $l$ stands for nullity. In this case $n = 6$ as $A^* \in F^{8 \times 6}$, we have the nullity being $l = n - r = 6 - 5 = 1$.

As $A^* = V \Sigma U^*$, we have $\{u_{r+1}, \dots, u_n \} = \{u_6 \}$ to be the basis of $null(A^*)$. As we have $null(A^*) \subset span(u_6)$ because they are both $\in F^m$. We also have $span(u_6) \subset null(A^*)$ as:

\begin{align*}
    A^* U &= V \Sigma \\
    A^* u_i &= \sigma_i v_i \ \text{for} \ i = r+1, \dots, n\\
    A^* u_i &= 0 \\
    \Longrightarrow& span(u_{r+1}, \dots, u_n) \subset null(A^*)
\end{align*}

Which is equivalent to $span(u_6) \subset null(A^*)$. Combine the two findings we have $span(u_6) = null(A^*)$. As $\{u_6 \}$ is obviously linearly independent, $\{u_6 \}$  is the basis of $null(A^*)$.

\end{document}

