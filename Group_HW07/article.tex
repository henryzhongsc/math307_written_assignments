\documentclass[11pt]{article}
\usepackage{setspace}
\setstretch{1}
\usepackage{amsmath,amssymb, amsthm}
\usepackage{graphicx}
\usepackage{bm}
\usepackage[hang, flushmargin]{footmisc}
\usepackage[colorlinks=true]{hyperref}
\usepackage[nameinlink]{cleveref}
\usepackage{footnotebackref}
\usepackage{url}
\usepackage{listings}
\usepackage[most]{tcolorbox}
\usepackage{inconsolata}
\usepackage[papersize={8.5in,11in}, margin=1in]{geometry}
\usepackage{float}
\usepackage{caption}
\usepackage{esint}
\usepackage{url}
\usepackage{enumitem}
\usepackage{subfig}
\usepackage{wasysym}
\newcommand{\ilc}{\texttt}
\usepackage{etoolbox}
\usepackage{algorithm}
\usepackage{changepage}
% \usepackage{algorithmic}
\usepackage[noend]{algpseudocode}
\usepackage{tikz}
\usepackage{gensymb}
\usetikzlibrary{matrix,positioning,arrows.meta,arrows}
\patchcmd{\thebibliography}{\section*{\refname}}{}{}{}
% \PassOptionsToPackage{hyphens}{url}\usepackage{hyperref}

\providecommand{\myceil}[1]{\left \lceil #1 \right \rceil }
\providecommand{\myfloor}[1]{\left \lfloor #1 \right \rfloor }
\providecommand{\qbm}[1]{\begin{bmatrix} #1 \end{bmatrix}}
\providecommand{\qpm}[1]{\begin{pmatrix} #1 \end{pmatrix}}
\providecommand{\norm}[1]{\left\lVert #1 \right\rVert}
\providecommand{\len}[1]{\left| #1 \right|}

\begin{document}




\title{\textbf{MATH 307: Group Homework 7}}


\author{\textit{Group 8}\\
Shaochen (Henry) ZHONG, Zhitao (Robert) CHEN, John MAYS, Huaijin XIN\\ \ilc{\{sxz517, zxc325, jkm100, hxx200\}@case.edu}}

\date{Due and submitted on 04/02/2021 \\ Spring 2021, Dr. Guo}
\maketitle




\subsection*{Problem 1}
\textit{See HW instruction.}\newline

\textbf{W.T.S} For all matrices $A$ in $F^{m \times n}$ with all of its row sums being $1$, the vector $\qbm{1 \\ 1\\ \vdots \\ 1} \in F^{n \times 1}$ can be the eigenvector.

\begin{align*}
    Av &= \lambda v \\
    \qbm{a^*_1 \\ a^*_2 \\ \vdots \\ a^*_n} \qbm{1 \\ 1 \\ \vdots \\ 1} &= \lambda  \qbm{1 \\ 1 \\ \vdots \\ 1} \\
    \text{Suppose} \ \lambda &= 1 \ \text{ we have} \\
    \qbm{1 \\ 1 \\ \vdots \\ 1} &=  \qbm{1 \\ 1 \\ \vdots \\ 1}
\end{align*}

Since the $Av = \lambda v$ equality is fulfilled with $\lambda = 1$, this is the corresponding eigenvalue for $v$.

\subsection*{Problem 2}
\textit{See HW instruction.}\newline

\subsubsection*{(a)}

\begin{align*}
    Au &= \lambda u \\
    (\lambda + c)u &= \lambda u + cu \\
    (A + cI)u &= Au + cIu = Au + cU = \lambda u + cu \\
    (A + cI)u &= (\lambda + c)u
\end{align*}

\subsubsection*{(b)}

In $\textbf{Problem 1}$ we have proved that for a matrix with row sum $1$, we may have $\qbm{1 \\ 1\\ \vdots \\ 1}$ to be the eigenvector. This conclusion can in fact be expand to matrices with row sum $s$, as:

\begin{align*}
    Av &= \lambda v \\
    A \qbm{1 \\ 1\\ \vdots \\ 1} &= \qbm{s \\ s\\ \vdots \\ s} = s \qbm{1 \\ 1\\ \vdots \\ 1} \\
    \lambda &= s \\
    v &= \qbm{1 \\ 1\\ \vdots \\ 1}
\end{align*}

Thus, for matrix with row sum $s = 0$ we have the eigenvalue being $\lambda = s = 0$.

Known that $Av = \lambda v = 0 v = 0$. For $A$ to be invertable we must have $AB = BA = I$, which implies there should be $BAv = Iv = 0 \Longrightarrow v = 0$. This contradicts $v = \qbm{1 \\ 1\\ \vdots \\ 1}$ and therefore $A$ is not invertible.


\subsection*{Problem 3}
\textit{See HW instruction.}\newline

Known that $Q$ is an unitary matrix, we must have $Q^* Q = QQ^* = I$. Now to investigate its eigenvalue $\lambda$:

\begin{align*}
    Qv &= \lambda v \\
    (Qv)^* &= (\lambda v)* \\
    v^* Q^* &=  \lambda^* v^*\\
    (v^* Q^*) Qv &= (\lambda^* v^*)  \lambda v \\
    v^* I v &= (\lambda^* \lambda) v^* v \\
    \norm{v}^2 &= \len{\lambda}^2 \norm{v}^2 \\
    1 &= \len{\lambda}^2 \\
    \len{\lambda} &= 1
\end{align*}

The equality in question is therefore proven.

\end{document}

