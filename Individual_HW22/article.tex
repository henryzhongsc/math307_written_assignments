\documentclass[11pt]{article}
\usepackage{setspace}
\setstretch{1}
\usepackage{amsmath,amssymb, amsthm}
\usepackage{graphicx}
\usepackage{bm}
\usepackage[hang, flushmargin]{footmisc}
\usepackage[colorlinks=true]{hyperref}
\usepackage[nameinlink]{cleveref}
\usepackage{footnotebackref}
\usepackage{url}
\usepackage{listings}
\usepackage[most]{tcolorbox}
\usepackage{inconsolata}
\usepackage[papersize={8.5in,11in}, margin=1in]{geometry}
\usepackage{float}
\usepackage{caption}
\usepackage{esint}
\usepackage{url}
\usepackage{enumitem}
\usepackage{subfig}
\usepackage{wasysym}
\newcommand{\ilc}{\texttt}
\usepackage{etoolbox}
\usepackage{algorithm}
\usepackage{changepage}
% \usepackage{algorithmic}
\usepackage[noend]{algpseudocode}
\usepackage{tikz}
\usepackage{pifont}
\usepackage{gensymb}
\usetikzlibrary{matrix,positioning,arrows.meta,arrows}
\patchcmd{\thebibliography}{\section*{\refname}}{}{}{}
% \PassOptionsToPackage{hyphens}{url}\usepackage{hyperref}

\providecommand{\myceil}[1]{\left \lceil #1 \right \rceil }
\providecommand{\myfloor}[1]{\left \lfloor #1 \right \rfloor }
\providecommand{\qbm}[1]{\begin{bmatrix} #1 \end{bmatrix}}
\providecommand{\qpm}[1]{\begin{pmatrix} #1 \end{pmatrix}}
\providecommand{\norm}[1]{\left\lVert #1 \right\rVert}
\providecommand{\len}[1]{\left| #1 \right|}
\newcommand{\cmark}{\ding{51}}%
\newcommand{\xmark}{\ding{55}}%

\begin{document}



\title{\textbf{MATH 307: Individual Homework 22}}


\author{Shaochen (Henry) ZHONG, \ilc{sxz517@case.edu}}

\date{Due on 05/03/2021 with extension granted till 05/06/2021, submitted on 05/03/2021 \\ Spring 2021, Dr. Guo}
\maketitle



\subsection*{Problem 1}
\textit{See HW instruction.}\newline

\noindent\textbf{W.T.S.}  $\sigma_i \in \Sigma$ are nonzero $\Longrightarrow$ $A$ is invertible.

For $A$ to be not invertible we must have $\det(A) = 0$, which implies:

\begin{equation*}
    \det(A) = \det(U \Sigma V^T) = \det(u) \det(\Sigma) = \det(V^T) = 0 \\
\end{equation*}

Since $U, V$ are orthognal matrices, we must have $\pm 1$ to be their determinants\footnote{because $\det(Q Q^T) = \det(I) \Longrightarrow \det(Q) \det(Q^{-1}) = 1 \Longrightarrow \det(Q)^2  =  1 \Longrightarrow \det(Q) = \pm 1$, assume $Q$ is a orthognal matrix.}. So we must have $\det(\Sigma) = 0$ for $A$ to be not invertible.

Also since $\Sigma$ is a diagonal matrix, we have $\det(\Sigma)$ to be the product of its diagonal valus. So for $\det(\Sigma) = 0$ we must have at least on of its singular value to be zero. So by contrapositive, we have showed that for $A$ to be invertible, there must be  $\sigma_i \in \Sigma$ are nonzero.\newline

\noindent\textbf{W.T.S.} $A$ is invertible $\Longrightarrow$ $\sigma_i \in \Sigma$ are nonzero.

The reverse is rather simple. Again by contrapositive, if $\Sigma$ has zero singular values, we have $\det(\Sigma) = 0$ and therefore $\det(A) = 0$, thus $A$ is not invertible by the determinant definition.\newline

\noindent As both directions are showed, the proposed iff statmenet is therefore proven.

\subsection*{Problem 2}
\textit{See HW instruction.}\newline

\begin{align*}
    \det(A - \lambda I) &= p(\lambda) = (\lambda_1 - \lambda)(\lambda_2 - \lambda)(\lambda_3 - \lambda) \dots (\lambda_4 - \lambda) \\
    &\text{Set} \ \lambda = 0 \\
    \det(A) &= \lambda_1 \lambda_2 \lambda_3 \dots \lambda_n
\end{align*}


\subsection*{Problem 3}
\textit{See HW instruction.}\newline




\end{document}

